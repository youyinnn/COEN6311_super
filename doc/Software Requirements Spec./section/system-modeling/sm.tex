\section{System Modeling \& Design}

\subsection{Use Cases Diagram}

\textbf{Overall Use Cases: }The overall use case is shown as Figure \ref{fig:oucs}.

\begin{figure*}[!ht]
	\centering
	\includegraphics[width=0.9\textwidth]{./img/Figure_overall-use-cases.jpg}
	\caption{Overall Use Cases}
	\label{fig:oucs}
\end{figure*}

\textbf{Use cases Diagram: }According to the various sub-requirements level,
we specify the use cases as follows of user requirements and system requirements level,
which are shown in different subgraphs of Figure \ref{fig:subusercases-1} and Figure \ref{fig:subusercases-2}.

\begin{figure*}[!ht]
	\centering
	\subfloat[SRS 1. Use Cases \\ Login Module \& User Info. Module]{\includegraphics[width=0.5\textwidth]{./img/SRS1.png}}
	% \hspace{0.5cm}
	\subfloat[SRS 2. Use Cases \\ Paper Download \& Paper Operation Module]{\includegraphics[width=0.5\textwidth]{./img/SRS2.png}}

	\subfloat[SRS 3. Use Cases \\ Paper Search Module]{\includegraphics[width=0.7\textwidth]{./img/SRS3.png}}
	\caption{Use case Diagrams for SRS 1, 2, 3}\label{fig:subusercases-1}
\end{figure*}

\begin{figure*}[!ht]
	\centering

	\subfloat[SRS 4. Use Cases \\ ICDE Record Capture Module \& ICDE Record Access Module]{\includegraphics[width=0.5\textwidth]{./img/SRS4.png}}
	% \hspace{0.5cm}
	\subfloat[SRS 5. Use Cases \\ ICDE Authorization Module]{\includegraphics[width=0.5\textwidth]{./img/SRS5.png}}

	\subfloat[SRS 6. Use Cases \\ Team Management Module]{\includegraphics[width=0.5\textwidth]{./img/SRS6.png}}
	% \hspace{0.5cm}
	\subfloat[SRS 7. Use Cases \\ Team Share Module]{\includegraphics[width=0.5\textwidth]{./img/SRS7.png}}

	\caption{Use case Diagrams for SRS 4, 5, 6,7}\label{fig:subusercases-2}
\end{figure*}

\subsection{System Business Objects Class Diagram}

According to the latest system requirement specifactions(SRS),
the system will have the following bussiness objects(BO) which serve the MVC.

% http://bib-it.sourceforge.net/help/fieldsAndEntryTypes.php
Figure \ref{fig:bo_classes_1} shows the very basic BOs of the system.
\textbf{Record} will be the topmost general BO above any other BO.
\textbf{Researcher} will represent the solid system user.
\textbf{PaperMetadata} will represent paper informations.
More fields of PaperMetadata can be refer to:
\url{http://bib-it.sourceforge.net/help/fieldsAndEntryTypes.php}.


\begin{figure*}[t]
	\centering
	\includegraphics[width=\textwidth]{./img/bo_classes_1.png}
	\caption{Bussiness Objects Part1: Record, Researcher, PaperMetadata}

	\label{fig:bo_classes_1}
\end{figure*}


Figure \ref{fig:bo_classes_2} shows the BOs which represent the users' preference about certain paper.
\textbf{PaperComment} will be the comment wrote by the user.
\textbf{PaperLikeAndDislike} will represent the preference from the user.

\begin{figure*}[t]
	\centering
	\includegraphics[width=0.7\textwidth]{./img/bo_classes_2.png}
	\caption{Bussiness Objects Part2: PaperComment, PaperLikeAndDislike}

	\label{fig:bo_classes_2}
\end{figure*}

Figure \ref{fig:bo_classes_3} shows the BOs which represent the users' paper collection.
\textbf{PaperCollectionCat} will be the collections which are created by the user.
\textbf{PaperCollectionRecord} will represent the single paper that collected within certain collection category.

\begin{figure*}[t]
	\centering
	\includegraphics[width=0.7\textwidth]{./img/bo_classes_3.png}
	\caption{Bussiness Objects Part3: PaperCollectionCat, PaperCollectionRecord}

	\label{fig:bo_classes_3}
\end{figure*}

Figure \ref{fig:bo_classes_4} shows the BOs which will gain the benefits from the ICDE.
\textbf{UserOperationRecord} will be the records that are created by the system whenever the system user performs paper-relevant actions.
\textbf{PaperClickTrending} will represent the trending of how many clicks on certain papers,
this data will generate by the system using the data recorded by ICDE.
\textbf{SearchTermTrending} will represent the trending of the hottest search terms.

\begin{figure*}[t]
	\centering
	\includegraphics[width=0.85\textwidth]{./img/bo_classes_4.png}
	\caption{Bussiness Objects Part4: UserOperationRecord,\\ PaperClickTrending, SearchTermTrending}

	\label{fig:bo_classes_4}
\end{figure*}


Figure \ref{fig:bo_classes_5} shows the BOs which represent Team features of the system.
\textbf{ResearchTeam} will represent a research team.
\textbf{ResearchTeamAuthRecord} will represent the relationship between users and teams.

\begin{figure*}[t]
	\centering
	\includegraphics[width=0.75\textwidth]{./img/bo_classes_5.png}
	\caption{Bussiness Objects Part5: ResearchTeam, ResearchTeamAuthRecord}

	\label{fig:bo_classes_5}
\end{figure*}

Figure \ref{fig:bo_classes_6} shows the BOs which will serve the 3rd-party ICDE application.
\textbf{ICDEThirdPartyApplication} will represent a registered 3rd-party ICDE application.
\textbf{ICDEThirdPartyAppAuthRecord} will represent the relationship between ICDE applications and users.

\textbf{$\ast$Notice} that BOs with {\color{OrangeRed} red box} are designed for future development of this system. It might not be implemented within the scope of the class project but it will still have classes or interfaces defined.

\begin{figure*}[t]
	\centering
	\includegraphics[width=0.75\textwidth]{./img/bo_classes_6.png}
	\caption{Bussiness Objects Part5: ICDEThirdPartyApplication, ICDEThirdPartyAppAuthRecord}

	\label{fig:bo_classes_6}
\end{figure*}

\subsection{Diagrams over Representatve Functions}

For SRS1.1.1-Register, it can be represent as activity diagram shown in Figure \ref{fig:srs_diagram_1}.

\begin{figure*}[t]
	\centering
	\includegraphics[width=0.5\textwidth]{./img/srs_diagram_1.png}
	\caption{Activity Diagram for SRS1.1.1}
	\label{fig:srs_diagram_1}
\end{figure*}

For SRS1.1.2-Update User informations, it can be represent as activity diagram shown in Figure \ref{fig:srs_diagram_2}.

\begin{figure*}[t]
	\centering
	\includegraphics[width=0.6\textwidth]{./img/srs_diagram_2.png}
	\caption{Activity Diagram for SRS1.1.2}
	\label{fig:srs_diagram_2}
\end{figure*}

For SRS1.2.1-Login, it can be represent as activity diagram shown in Figure \ref{fig:srs_diagram_3}.

\begin{figure*}[t]
	\centering
	\includegraphics[width=0.5\textwidth]{./img/srs_diagram_3.png}
	\caption{Activity Diagram for SRS1.2.1}

	\label{fig:srs_diagram_3}
\end{figure*}


For SRS2.1.1-Download Paper, it can be represent as sequence diagram shown in Figure \ref{fig:srs_diagram_4}.

\begin{figure*}[t]
	\centering
	\includegraphics[width=1\textwidth]{./img/srs_diagram_4.png}
	\caption{Sequence Diagram for SRS2.1.1}

	\label{fig:srs_diagram_4}
\end{figure*}


For SRS 2.1.2, SRS 2.2.4, SRS 4.2, and SRS 5, they can be represent as activity diagram shown in Figure \ref{fig:srs_diagram_5}.

\begin{figure*}[t]
	\centering
	\includegraphics[width=0.6\textwidth]{./img/srs_diagram_5.png}
	\caption{Activity Diagram for SRS 2.1.2, SRS 2.2.4, SRS 4.2, and SRS 5}

	\label{fig:srs_diagram_5}
\end{figure*}

For SRS 6.1.2-Team Invitation, it can be represent as activity diagram shown in Figure \ref{fig:srs_diagram_6}.
Also it indicates how ICDE functions. And its sequence diagram is shown in Figure \ref{fig:srs_diagram_7}.
And also its state diagram shown in Figure \ref{fig:srs_diagram_8}

\begin{figure*}[t]
	\centering
	\includegraphics[width=0.8\textwidth]{./img/srs_diagram_6.png}
	\caption{Activity Diagram for SRS 6.1.2}

	\label{fig:srs_diagram_6}
\end{figure*}

\begin{figure*}[t]
	\centering
	\includegraphics[width=\textwidth]{./img/srs_diagram_7.png}
	\caption{Sequence Diagram for SRS 6.1.2}

	\label{fig:srs_diagram_7}
\end{figure*}

\begin{figure*}[t]
	\centering
	\includegraphics[width=0.8\textwidth]{./img/srs_diagram_8.png}
	\caption{State Diagram for SRS 6.1.2}

	\label{fig:srs_diagram_8}
\end{figure*}
