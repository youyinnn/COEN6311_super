\section{User Scenario \& User Story}

\noindent
The system has three main scenarios,
Table \ref{table:us1}, table \ref{table:us2}, and table \ref{table:us3} present three user scenarios
for better understanding of the system. And their corresponding user stories are given following the table.

\subsection*{Scenario One}

\begin{table}[!ht]
	\renewcommand{\arraystretch}{1.2}
	\centering
	\begin{tabular}{ p{14cm} }
		\hline
		\textbf{Researcher}                                                            \\
		\hline
		\textbf{Jack} is a new university student. He got an individual assignment last week
		from one of the courses which require  academic research. With the assignment demand,
		Jack should search at least 3 to 4 academic papers for investigation. So he heads to
		the \textbf{ICDE-ScholarHub} website.                                          \\
		\\
		He goes to the website and registers an account of it by filling in all the information
		required by the website.                                                       \\
		\\
		After finished registering, he then type some keywords of the subject from the assignment.
		The website displays all the results that match the keywords.                  \\
		\\
		He can read some information about the paper on its detail page. He will know the authors,
		the year of the publication, the academic area, and the abstraction. He will know that
		how many people are like this paper, what are the comments of it.              \\
		\\
		Then he clicks the "download" button and then gets the pdf file of this paper. \\
		\\
		After finish the reading of it, he then login to the website, writes a comment about it
		and clicks the "dislike" button to express his idea of this paper.             \\
		\hline
	\end{tabular}
	\caption{New user's story}
	\label{table:us1}
\end{table}

\begin{itemize}
	\item [1] Jack wants to register a user account, so he can log in to the system.
	\item [2] Jack wants to log out from the system, so he would not leave my personal information on the system.
	\item [3] Jack wants to search papers with certain keywords, so he can browse the result for his assignment.
	\item [4] Jack wants to see the detail page of the paper that he clicks into, so he will know more about the paper.
	\item [5] Jack wants to comment on the paper, so he can express his idea of it.
	\item [6] Jack wants to download the paper, so he can read the pdf file of the paper.
	\item [7] Jack wants to click the "dislike" button, so the other will know it.
\end{itemize}

\subsection*{Scenario Two}

\begin{table}[!ht]
	\renewcommand{\arraystretch}{1.2}
	\centering
	\begin{tabular}{ p{14cm} }
		\hline
		\textbf{Researcher}                                                      \\
		\hline
		\textbf{Lucy} is a Ph.D. student researching a program for several months. She likes to
		use the system for paper searching.                                      \\
		\\
		After work, she will check the paper trending list once to twice a season. With
		this trending list, it is easy to notice what are the most popular topics and
		papers people like during the past season.                               \\
		\\
		She will know how many people have visited the detail page of the paper. She will
		know how many new "like" have been given for the paper from other users. \\
		\\
		And when she is at work, she can see her paper browsing history on the website
		and then find the one she wants to read again.                           \\
		\hline
	\end{tabular}
	\caption{Common user's story}
	\label{table:us2}
\end{table}

\begin{itemize}
	\item [1] Lucy wants to see the paper trending board, so she can know how is it like of current academic trend.
	\item [2] Lucy wants to see her paper browsing history, so she can retrieve her previous work.
\end{itemize}

\subsection*{Scenario Three}

\begin{table}[!ht]
	\renewcommand{\arraystretch}{1.2}
	\centering
	\begin{tabular}{ p{14cm} }
		\hline
		\textbf{Research Team}                                                               \\
		\hline
		\textbf{Tom} and \textbf{Kail} are teammates within the same research team. They are
		working on the same project with other professors and schoolmates.                   \\
		\\
		The whole team is on the same team created by the ScholarHub system. Tom is the
		administrator of the team. He can add or remove a member to the team's member board. \\
		\\
		During their work, Tom can see other team members' activities such as what paper are
		they currently reading or what is the attitude of the member to a particular paper.  \\
		\\
		Also, every member of the team can share the paper they think is helpful for the team's
		research so that the others will know about this paper.                              \\
		\hline
	\end{tabular}
	\caption{Team users' story}
	\label{table:us3}
\end{table}

\begin{itemize}
	\item [1] Tom wants to create a research team on the system, so he can share information with other teammate.
	\item [2] Tom wants to add Kail to the research team, so they can share more information with each other.
	\item [3] Tom wants to remove a teammate, so the whole team will not get the information from this person.
	\item [4] Tom wants to share a paper with other teammates on the system team page, so they can get it ASAP.
	\item [5] Kail wants to join Tom's team, so he can work with the team.
\end{itemize}
